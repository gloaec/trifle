\thesistitle{
    Élaboration d'un système de gestion de configuration sensible au contexte
    pour un réseau d'applications.
    %Modèle distribué d’abstraction des informations de contexte, données de
    %configuration et politiques d’administration d’un système de gestion de
    %configuration sensible au contexte dans un reseau d’applications
}

\degreename{Master Recherche Image Informatique et Ingénierie}
\degreefield{Informatique}

\authorname{Ghislain Loaec}
\committeechair{Nader Mbarek}
\othercommitteemembers{Emmanuel Garette}

\degreeyear{2014}
\copyrightdeclaration {
    {\copyright} {\Degreeyear} \Authorname
}

\acknowledgments {
  Je tiens tout d'abord à remercier mon tuteur M. Nader Mbarek pour m'avoir fait
  confiance, puis pour m'avoir guidé, encouragé et conseillé tout en me laissant
  une grande liberté dans la définition et les évolutions du sujet choisi.  Mes
  remerciements vont également à M. Albert Dipanda sans qui ce stage n'aurait
  jamais été possible. Et encore merci à ces messieurs, qui ont accepté d'être
  les rapporteurs de ce mémoire et je les en remercie, de même que pour leur
  participation au Jury.

  Je ne sais comment exprimer ma gratitude à tous les salariés de la société
  Cadoles pour m'avoir permit de menerà bien ce stage dans les meilleurs
  conditions. Je remercie tout particulièrement M. Emmanuel Garette, gérant de
  la société, pour m'avoir suivi tout au long de ce stage et pour la gentillesse
  et la patience qu'il a manifestées à mon égard, ainsi que pour la pertinence
  de ses suggestions.

  Je remercie tous ceux sans qui ce mémoire ne serait pas ce qu'il est :
  aussi bien par les discussions que j'ai eu la chance d'avoir avec eux, leurs
  suggestions ou contributions. Je pense ici en particulier à Daniel Dehennin
  pour les conseils stimulants que j'ai eu le plaisir de recevoir et les
  orientations qu'il a donné à mes lectures.

  Pour leurs encouragements et leur assistance morale, je remercie chaudemment
  ma famille et mes amis. Merci à Mlle. Camille Sintive pour avoir accepté
  d'effectuer un travail de relecture et qui par ses nombreuses remarques et
  suggestions a permis d'améliorer la qualité de ce mémoire, je lui en suis très
  reconnaissant.

  Je remercie de plus tous les auteurs des programmes du domaine public que j'ai
  intensément utilisés durant ce stage, à savoir tous les contributeurs à \TeX,
  \LaTeX, linux, xfig, Zotero et netlib. Sans eux, mes conditions de travail
  auraient sans doute été très différentes et beaucoup moins agréables. Bien que
  je ne les aie jamais rencontrés je remercie aussi E.Summers, D.Kresh et
  G.Higgins, et d'autres dont je ne connais pas le nom, car j'ai profité de leur
  librairie RDFlib qui m'a beaucoup facilité l'approche des ontologies en
  langage Python. Je remercie également M. Lars Otten dont j'ai pillé
  allègrement les formats \LaTeX.

  Enfin, ces remerciements ne seraient pas complets sans mentionner M. Mark
  Burgess, à qui la majorité du travail sur la configuration autonome est due
  qui m'a largement rendu grâce à sa profonde érudition sur le sujet.
}

\newcommand{\mypubentry}[3]{
  \begin{tabular*}{1\textwidth}{@{\extracolsep{\fill}}p{4.5in}r}
    \textbf{#1} & \textbf{#2} \\ 
    \multicolumn{2}{@{\extracolsep{\fill}}p{.95\textwidth}}{#3}\vspace{6pt} \\
  \end{tabular*}
}

\newcommand{\mysoftentry}[3]{
  \begin{tabular*}{1\textwidth}{@{\extracolsep{\fill}}lr}
    \textbf{#1} & \url{#2} \\
    \multicolumn{2}{@{\extracolsep{\fill}}p{.95\textwidth}}
    {\emph{#3}}\vspace{-6pt} \\
  \end{tabular*}
}

\thesisabstract {
  L'objectif fondamental de l'informatique ubiquitaire est de faciliter
  l'utilisation de l'ordinateur. Cela passe par extraire le maximum de bénéfices
  de l'environement numérique. Les défaillances logicielles deviennent monnaie
  courante à mesure que les systèmes informatiques et leur complexité continuent
  de croître. Le problème réside principalement dans l'absence de standards ou
  de modèles réutilisables pour la gestion des informations de contexte. Ce
  mémoire a pour objectif d'apporter un approche simplifiée de la gestion de la
  configuration dans une infrastructure orientée sur les services. Cette
  approche conjugue des concepts tels que les ontologies, la théorie de la
  promesse et le consensus de Raft pour aboutir à un système multi-agents
  tolérant à l'échec.  Les ontologies possèdent un haut degré de formalisme et
  serviront à modéliser les informations de contexte et les politiques de
  comportement. La théorie de la promesse ouvre de nouvelles perspectives sur la
  façon de gérer la configuration, notamment par l'introduction d'une dimension
  sémantique dans la définition de règles de gestion. Enfin l'algorithme de
  consensus de Raft permettra d'abstraire complètement les aspects de
  synchronistation entre les agent, de réplication de l'information et garantit
  une information toujours consistante.
}

% ex: set spelllang=fr spell: %
%%% Local Variables: ***
%%% mode: latex ***
%%% TeX-master: "thesis.tex" ***
%%% End: ***

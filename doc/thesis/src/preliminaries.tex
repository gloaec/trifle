\thesistitle{Modèle d'abstraction des données de contexte dans la configuration
d'un reseau d'applications}
\degreename{Master Recherche Image Informatique et Ingénierie}
\degreefield{Informatique}
\authorname{Ghislain Loaec}
\committeechair{Nader Mbarek}
\othercommitteemembers{Emmanuel Garette}
\degreeyear{2014}
\copyrightdeclaration {
  {\copyright} {\Degreeyear} \Authorname
}

\acknowledgments {
  Je souaiterais remercier...
}

\newcommand{\mypubentry}[3]{
  \begin{tabular*}{1\textwidth}{@{\extracolsep{\fill}}p{4.5in}r}
    \textbf{#1} & \textbf{#2} \\ 
    \multicolumn{2}{@{\extracolsep{\fill}}p{.95\textwidth}}{#3}\vspace{6pt} \\
  \end{tabular*}
}

\newcommand{\mysoftentry}[3]{
  \begin{tabular*}{1\textwidth}{@{\extracolsep{\fill}}lr}
    \textbf{#1} & \url{#2} \\
    \multicolumn{2}{@{\extracolsep{\fill}}p{.95\textwidth}}
    {\emph{#3}}\vspace{-6pt} \\
  \end{tabular*}
}

\thesisabstract {
  L'objectif fondamental de l'informatique ubiquitaire est de faciliter
  l'utilisation de l'ordinateur. Cela passe par extraire le maximum de bénéfices
  de l'environement numérique. Les défaillances logicielles deviennent monnaie
  courante à mesure que les systèmes informatiques et leur complexité continuent
  de croître. Le problème réside principalement dans l'absence de standards ou
  de modèles réutilisables pour la gestion des informations de contexte.
}

% ex: set spelllang=fr spell: %
%%% Local Variables: ***
%%% mode: latex ***
%%% TeX-master: "thesis.tex" ***
%%% End: ***

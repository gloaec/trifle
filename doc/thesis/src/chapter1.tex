\chapter{État de l'art}

\section{Introduction}

La configuration des composantes logicielles impose un coût majeur dans
l'administration d'un système. Des erreurs de configuration peuvent se traduire
par des vulnérabilités en termes de sécurité, de sévères pertubations dans le
fonctionnement de la brique logicielle, ou purement et simplement provoquer un
déni de service. La prise en considération du contexte pourrait permettre une
abstraction partielle ou complète de cette couche très technique et extrêmement
pénible à configurer.

Un système sensible au contexte doit être capable de mimer la capacité humaine à
reconaître et exploiter l'information implicitement présente dans
l'environement. Cela implique une configuration dynamique de chacune des
composantes de l'architecture, de manière à pouvoir ajuster leur comportement
respectif en fonction de la situation. Identifier l'activité humaine est un
défi, il est essentiel que les applications opèrent en transmettant
l'information appropriée au bon endroit et au bon moment par inférence de
l'intention des utilisateurs. L'infomatique sensible au contexte est un
paradigme dans lequel les application peuvent découvrir et tirer profit d'
informations de circonstance telles que la position actuelle, l'heure de la
journée, les personnes et périphériques dans l'environement et leurs activités.

Dans ce mémoire, nous aborderons les principes communs à chacunes des
architectures existantes, desquels nous detaillerons le framework conceptuel
dérivé (!) par couches. Nous présenterons un certaine variété d'intergiciels et
d'infrastructures reconnus pour faciliter la configuration d'applications et de
services basés sur le contexte \cite{uci-thesis-latex}.

% This is an example using the \LaTeX{} template for UCI theses and
% dissertation documents \cite{uci-thesis-latex}. Figure
% \ref{fig:sourcecode} is just for illustration purposes, as is Table
% \ref{tab:coordinates}.
% 
% \begin{figure}
% \begin{verbatim}
% #include <iostream>
% int main(int argc, char** argv) {
%   std::cout << "Hello World." << std::endl;
%   return 0;
% }
% \end{verbatim}
%   \caption{Example source code.}
%   \label{fig:sourcecode}
% \end{figure}

\section{Background}

% Lorem ipsum dolor sit amet, consectetur adipisicing elit, sed do
% eiusmod tempor incididunt ut labore et dolore magna aliqua. Ut enim ad
% minim veniam, quis nostrud exercitation ullamco laboris nisi ut
% aliquip ex ea commodo consequat. Duis aute irure dolor in
% reprehenderit in voluptate velit esse cillum dolore eu fugiat nulla
% pariatur. Excepteur sint occaecat cupidatat non proident, sunt in
% culpa qui officia deserunt mollit anim id est laborum.
% 
% \begin{table}
%   \centering
%   \begin{tabular}{|rr|r|}
%     \hline
%     $x$ & $y$ & $z$ \\
%     \hline
%     14 & 12 & -2 \\
%     0 & 33 & -25 \\
%     -3 & 11 & 22 \\
%     4 & 4 & 6 \\
%     \hline
%   \end{tabular}
%   \caption{Example coordinates.}
%   \label{tab:coordinates}
% \end{table}
% 
% Lorem ipsum dolor sit amet, consectetur adipisicing elit, sed do
% eiusmod tempor incididunt ut labore et dolore magna aliqua. Ut enim ad
% minim veniam, quis nostrud exercitation ullamco laboris nisi ut
% aliquip ex ea commodo consequat. Duis aute irure dolor in
% reprehenderit in voluptate velit esse cillum dolore eu fugiat nulla
% pariatur. Excepteur sint occaecat cupidatat non proident, sunt in
% culpa qui officia deserunt mollit anim id est laborum.


%%% Local Variables: ***
%%% mode: latex ***
%%% TeX-master: "thesis.tex" ***
%%% End: ***

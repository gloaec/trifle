\chapter{État de l'art}

\section{Introduction}

La configuration des composantes logicielles impose un coût majeur dans
l'administration d'un système. Des erreurs de configuration peuvent se traduire
par des vulnérabilités en termes de sécurité, de sévères pertubations dans le
fonctionnement de la brique logicielle, ou purement et simplement provoquer un
déni de service. La prise en considération du contexte pourrait permettre une
abstraction partielle ou complète de cette couche très technique et extrêmement
pénible à configurer.

Un système sensible au contexte doit être capable de mimer la capacité humaine à
reconaître et exploiter l'information implicitement présente dans
l'environement. Cela implique une configuration dynamique de chacune des
composantes de l'architecture, de manière à pouvoir ajuster leur comportement
respectif en fonction de la situation. Identifier l'activité humaine est un
défi, il est essentiel que les applications opèrent en transmettant
l'information appropriée au bon endroit et au bon moment par inférence de
l'intention des utilisateurs. L'infomatique sensible au contexte est un
paradigme dans lequel les application peuvent découvrir et tirer profit d'
informations de circonstance telles que la position actuelle, l'heure de la
journée, les personnes et périphériques dans l'environement et leurs activités.

Dans ce mémoire, nous aborderons les principes communs à chacunes des
architectures existantes, desquels nous detaillerons le framework conceptuel
dérivé (!) par couches. Nous présenterons un certaine variété d'intergiciels et
d'infrastructures reconnus pour faciliter la configuration d'applications et de
services basés sur le contexte.

\section{Background}

De nombreux débats ont eu lieu sur .. Alors que la plupart des gens comprènent
de manière tacite ce qu'est le contexte, ils le trouve par ailleurs
particulièrement difficile à élucider.

\begin{figure}[h]
  \centering
  \begin{tabular}{l}
    ``... toute information pouvant être utilisée pour caractériser la
    situation d'une entité.\\
    Une entité peut être une personne, un lieu ou un
    objet considéré pertinent dans l'interaction \\
    entre un utilisateur et une
    application, notamment l'utilisateur et l'application eux même.``
    \cite{abowd_baltzer_1997} \\
    \em \footnotesize Gregory D. Abowd, Christopher G. Atkeson, Jason Hong, Sue
    Long, Rob Kooper, and Mike \\
    \em \footnotesize Pinkerton. Cyberguide: a mobile context-aware tour guide.
    Wireless Networks, 3(5):421–433, October 1997. 
  \end{tabular}
  \caption{Le context défini par Abowd et. al.}
  \label{fig:quote}
\end{figure}

Cette définition rend la tâche plus facile à un développeur d'application pour
énumérer le contexte pour un scénario d'application donné. Si un fragment
d'information peut être utilisé pour caractériser la situation d'un participant
dans une quelquonque interaction, alors cette information appartient au
contexte.

\section{Vue d'ensemble sur le contexte}

\begin{figure}[h]
  \centering
  \begin{tabular}{l}
    ``Le contexte est efficace, seulement lorsqu'il est partagé.``
    \cite{winograd_architectures_2001} \\
    \em \footnotesize Terry Winograd. Architectures for Context. \\
    \em \footnotesize Human-Computer Interaction, 16(2):401–419,
     December 2001. \\
  \end{tabular}
  \caption{Winograd a propos du contexte}
  \label{fig:quote}
\end{figure}

Pour s'assurer que le contexte soit partagé, il doit d'abord être receilli et
rigoureusement traité. Cela implique qu'un système sensible au contexte doit
être en mesure de comprendre ce qu'est le contexte avant d'aller à la recherche
de ces informations et de pouvoir les catégoriser.

\subsection{Classes de contexte}

Schilit et. al. proposent la classification suivante des informations de
contexte:

\begin{itemize}
  \item \textbf{Contexte Informatique} - Connectivité réseau, bande passante,
	  and ressources à proximité telles que des imprimantes, des
	  affichages ou des postes de travail.
  \item \textbf{Contexte Utilistateur} - Le profil utilisateur, sa situation
	  géographique, sa situation sociale actuelle et les individus qui
	  l'entourent.
  \item \textbf{Contexte Physique} - L'éclairage, le niveau de bruit, les
	  condition de circulation ou la température.
\end{itemize}

Chacune de ces catégories contiennent une richesse d'informations pertinantes
pour le système sensible au contexte. Elle ne peuvent cependant pas être
traitées de manière isolée pour pouvoir en extraire le meilleur. L'intention du
système sensible au contexte est de rassembler et de fusionner ces informations
pour aboutir à une vue d'ensemble de la situation. Une fois le contexte mis en
tampon ou en base, le système doit alors filtrer les informations pertinantes
pour l'utilisateur, dans le moment présent.

Les informations de contexte peuvent alternativement être subdivisées en 2
catégories bien distinctes: contexte virtuel ou physique.

\subsubsection{Contexte virtuel}

Le contexte virtuel inclut la version du système d'exploitation, les
possibilités d'interface, la technologie en charge de l'accomplissement des
communications, les emails envoyés et reçus, et les documents édités.

\subsubsection{Contexte physique}

Les contexte physique d'un autre coté peut être la présence d'une autre entité,
qu'elle soit utilisateur ou périphérique, la proximité d'un imprimante en
particulier, une indication que l'utilisateur est debout, en train de marcher ou
assis ou les conditions météorologiques actuelles. En d'autres termes, le
contexte physique peut être défini comme toute donnée aquierable par le biais
d'une sonde.

\subsubsection{Contexte historique}

Les contextes mémorisés au cours d'un certain laps de temps. Cette information
est considérée très utile, mais n'est que très rarement utilisée, sauf pour les
applications mobiles. Le système doit être en mesure d'estimer les informations
valant la peine d'être conservées. Cette évaluation est excessivement coûteuse
et nécéssite donc des algorithmes très performants.

\subsection{Les caractéristiques des informations de contexte}

Les chercheurs l'université de Queensland ont classifié quatre caractéristiques
majeures : \cite{catharina_context_2002}

\begin{enumerate}
    \item \textbf{Les informations de contexte présente une gamme de
	    caractéristiques temporelles}

	    L'information de contexte est d'hors et déjà catégorisée selon
	    l'environement auquel il appartient : virtuel ou physique ; elle
	    peut en outre est subdivisée selon un critère de temporalité :

            \begin{itemize}
		\item Information statique : toute information apparentée à
			l'environement de l'utilisateur qui ne varie pas.
		\item Information dynamique : l'information accumulée
			continuellement, fréquemment et automatiquement.
            \end{itemize}

	    De plus, l'information de contexte passé semble indispensable pour
	    la compréhension de l'état global de l'environnement.

    \item \textbf{L'information de contexte n'est pas parfaite}

	    Cela considère la validité du contexte, majoritairement concernant
	    les informations de contexte dynamique. La vitesse et la fréquence à
	    laquelle l'information varie soulève de sérieuses raisons de
	    douter de sa solidité. Ce ``délai entre la production et
	    l'utilisation de l'information de contexte``
	    \cite{catharina_context_2002} est une préoccupation non-négligeable.
            D'autres sources d'inquiétudes quant au bien-fondé de
	    l'information de contexte incluent la fiabilité des informations
	    fournies par les producteurs de contexte : défaillance d'un capteur,
	    chemin rompu entre les producteurs ou n'importe quelle source qui
	    fournirait une information erronée ou désuette.

    \item \textbf{L'information incarne un grand nombre de représentations}

            Les données brutes recueillies depuis l'environemment physique et
	    virtuel peuvent prendre de nombreuses formes et doivent être
	    traitées pour se mêler à d'autres informations de contexte. La
	    probabilité pour qu'un système sensible au contexte obtienne un taux
	    de succès de 100\% lors d'une ``capture des relations existantes
	    entre les représentations alternatives``
	    \cite{catharina_context_2002}  de l'information et celle apte à la
	    situation courante est quasi-nulle.

    \item \textbf{Les informations de contexte sont très fortement correlées}

	    L'information contextuelle provenant d'une origine particulière peut
	    avoir un lien très étroit avec sa source, si bien qu'elle est
	    dépendante de l'origine.
	    Le contexte peut ne pas être fiable ''là où les caractéristiques de
	    l'information dérivée sont intimement liées aux propriétés de
	    l'information dont il est issu.'' \cite{catharina_context_2002}

\end{enumerate}

\subsection{Système sensible au contexte}

Un système est dit sensible au contexte s'il utilise le contexte pour fournir
des informations pertinentes et/ou des services à l'utilisateur, où la
pertinence dépend de la tâche de l'utilisateur. Les systèmes sensibles au
contexte peuvent être implémentés sous plusieures formes, mais pour accomplir
cet objectif, le système doit d'un manière générale :

\begin{itemize}
    \item Recceuillir l'information
    \item Sérialiser cette information
    \item Fusionner l'information pour générer un contexte de plus haut niveau
    \item Prendre automatiquement des mesures basées sur l'information
	    recceuillie
    \item Rendre l'information disponible à l'utilistateur, dans l'immédiat,
	    dans le futur ou au moment approprié pour améliorer et aider à la
	    completion de la tâche de l'utilisateur.
\end{itemize}

Les chercheurs du Context Toolkit à l'université de Berkley proposent 3
fonctionnalités qu'une application sensible au contexte doit impérativement
supporter : \cite{dey_providing_2000}

\begin{enumerate}
    \item Présentation de l'information et des services à l'utilistateur
    \item Éxécution automatique d'un service pour l'utilisateur
    \item Etiquetage de l'information de contexte pour une extraction ultérieure
\end{enumerate}

Les développeurs du systems Kimura ont conçu des composants distribués destinés
à un système sensible au contexte global. Ces composants se divisent en trois
classes \cite{voida_integrating_2002}, qui sont les suivantes :

\begin{enumerate}
    \item Aquisition du contexte - Le système récupère l'information de contexte
	    et l'ajoute dans un dépôt dédié.
    \item Interprétation du contexte - Les sytème convertit l'information
	    recceuillie en un contexte de travail.
    \item Interaction de l'utilisateur - Le système affiche le contexte de
	    travail à l'uilisateur.
\end{enumerate}

\subsection{Recceuillir l'information}

Certaines informations de contexte sont explicitement données au sytème, comme
le nom de l'utilisateur, son âge, son addresse email, les variables
d'environemment or le registre. D'autres informations physiques primitives comme
la lumière, la température ou des lectures de pression peuvent être acquises par
l'intérmédiaire d'un capteur.

La situation géographique et l'identité sont les deux fragments de contexte les
plus fréquemment détéctés.

\begin{figure}[h]
  \centering
  \begin{tabular}{l}
    ``Les capteurs ne sont pas toujours précis ou fiables à 100\%, \\
    tout particuièrement s'ils sont à usage unique. Le processus de \\
    récupération de l'information doit être tolérent à la défaillance \\
    protentielle d'un capteur. Toutes les informations recueillies doivent \\
    être assujetties à des contrôles de validité pour vérifier leur exactitude.\\
    La fusion des capteur est un moyen de remédier à cette difficulté.``
    \cite{schmidt_there_1999} \\
    \em \footnotesize Albrecht Schmidt, Michael Beigl, and Hans-W Gellersen. \\
    \em \footnotesize There is more to context than location. Computers \&
    Graphics, 23(6):893–901, 1999.
  \end{tabular}
  \caption{La fiabilité des capteurs par Schmidt}
  \label{fig:quote}
\end{figure}

Prenez l'exemple d'une centrale éléctrique, possédant de nombreuses sondes de
témpérature implantées à diverses endroits d'un reacteur nucléaire. Le système
doit prendre la déscision de refroidir plus ou moins généreusement en fonction
de ces informations sondées. La sous-estimation du contexte de température
aurait des conséquences catastrophiques. Il semblerait judicieux de simplement
calculer un valeur moyenne et/ou d'écarter les valeurs reportées qui
manifestemment diffèreraient trop des autres mesures. Cette méthode éviterait
des prises de mesures drastiques du système sesnsible au contexte dans le but de
corriger ce qu'il considère comme une variation de température, mais qui en
réalité n'est que le résultat d'une panne de capteur.

\subsection{Récupérer l'information}

\begin{itemize}
  \item Modèle ''Push'' - La source de contexte recupère l'information avant
	  même qu'elle soit requise. Cela améliore indéniablement les
	  performances, mais nécéssite une consommation des resources fréquente
	  et considérable pour une information qui pourrait bien ne jamais être
	  exploitée.
  \item Modèle ''Pull'' - Collecte l'information de contexte au moment opportin.
	  Cela autorise le recueil d'information à la demande, mais expose le
	  système aux latences réseaux et aux indisponibilités potentielles de
	  certains services.
\end{itemize}

La méthode de d'acquisiton des données de contexte est très importante lorsque
l'on conçoit un système de cette nature, puisque c'est ce qui prédéfinit le style
architectural du système. Chen (2003) \cite{chen_intelligent_2003} présente
trois approches différentes pour acquérir l'information de contexte.

\subsubsection{Capteur en accès direct}

Cette approche est souvent utilisée dans les periphériques avec capteurs
intégrés. Le logiciel client récupère l'information désirée directement depuis
les sondes, i.e., il n'y a pas de couche supplémentaire pour obtenir et traiter
les données du capteur. Les pilotes pour les capteurs sont raccordés à
l'application, donc cette méthode de couplage étroit n'est utilisable que dans
des cas très rares. Par conséquent, elle n'est pas adapté pour les sysèmes
distribués.

\subsubsection{Infrastructure intergicielle}

L'approche intergicielle introduit un architecture en couches avec l'intention
d'abstraire les détails de bas-niveau du processus de détection. La méthode est
équivalente au modèle client-serveur, plus flexible que le widget puisqu'elle
favorise l'indépendance de chacunes des composantes du système. Chaque
composante doit être en mesure d'effectuer les opération suivantes : établir des
connexions, envoyer et recevoir des messages et gérer les erreurs. Cet modèle
est plus complexe de manière significative, mais l'approche est plutôt aisée
puisqu'elle support un grand nombre de périphériques et d'applications pas
l'utilisation de normes de codage et des protocoles réseaux standards.

\subsubsection{Serveur de contexte}

Cette approche distribuée élargie l'infrastructure basée sur les intergiciels en
introduisant un gestionnaire d'accès distant. Les données recceuillies par les
capteur sont déplacées vers ce dit ''serveur de contexte'' dans le but de
faciliter les accès currents.

\subsection{Les architectures de gestion de contexte}

Winograde (2001) \cite{winograd_architectures_2001} décrit trois modèles
différents afin de coordiner les proccessus et composantes multiples :

\begin{itemize}
        \item \textbf{Widgets}
		L'objectif clé du widget est distinguer l'application du
		processus d'acquisition de contexte, de manière d'abstraire la
		complexité que représente le recueil et la gestion de
		l'information de contexte. Le widget est considéré comme un
		médiateur qui transmet exclusivement des informations
		pertinentes à l'appplication. Un widget de contexte fonctionne
		totalement indépendament de l'application, cet qui permet à
		plusieures applications d'en faire usage simultanément. Le
		widget est notamment responsable de l'entretien d'un historique
		complet du contexte sondé au fil du temps. C'est le modèle le
		plus répandu.

        \item \textbf{Services réseaux}
		Cette approche plus flexible, comme l'argumente Hong
		and Landay (2001) \cite{hong_infrastructure_2001}, ressemble à
		l'architecture de serveur de contexte. Au lieu d'un
		gestionnaire de widget centralisé, des techniques sont utilisées
		pour découvir les services réseaux dans l'infrastructure. Cette
		approche basée sur les services n'est aussi performante que
		l'architecture basée sur les widgets à cause de la complexité
		des composants orientés réseaux, mais fournit une certaine
		robustesse.

	\item \textbf{Tableau noir (Blackboard)}
                En contraste avec la vue orientée processus du widget et le
		modèle orienté services, le tableau noir représente un approche
		orientée sur les données. Cete approche donne un focus tout
		particulier aux données, correspondant des motifs spécifiques
		dans les données. Dans ce modèle asymétrique, les processus
		envoient de messages dans un média partagé, le dit ''tableau
		noir'', et souscrivent à recevoir des notifications lorsque
		certains évenements se produisent. Les avantages de ce modèles
		résident dans la simplicité de configuration et d'ajout de
		nouvelles sources de contexte.
\end{itemize}

\subsubsection{Critères d'arbitrage}

Afin de définir au mieux quel modèle choisir dans la mise en place d'un système
sensible au contexte, il est avisé de considérer les caractéristiques suivantes
:

\begin{itemize}
    \item \textbf{Efficacité} : 
	    accélérer le débit de l'information compte tenu de la bande passante
	    et de la latence causée par l'explosion du nombre d'applications et
	    de périphériques dans le réseau.
    \item \textbf{Effort de configuration} : 
	    compte tenu de la quantité variable de composants, effectuer des
	    changements dans l'état de la configuration, sans engendrer des
	    pertubations ou mettre le système en échec, n'est pas une tâche
	    fastidieuse. Le modèle doit s'assurer que l'édition est sans danger.
    \item \textbf{Robustesse} : 
	    Le degré auquel le système peut faire face à la défaillance
    \item \textbf{Simplicité} : 
            \begin{quotation}
              un système qui nécéssite une compréhension avancée des méchanismes 
	      qu'il implémente pour faire l'usage, ne sera utilisé que par ceux
	      qui auront le dévouement et la motivation de le maîtriser
	      \cite{winograd_architectures_2001}
            \end{quotation}
    \item \textbf{Extensibilité} : 
            \begin{quotation}
              Les services supportant la notion générale de contexte doivent
	      être facilement extensible pour accueillir toute nouvelle source
	      d'information de contexte non-anticipée. \cite{ebling_issues_2001}
            \end{quotation}
\end{itemize}

Ces critères peuvent être utilisés pour comparer et contraster les différents
modèles de gestion de contexte présentés précédemment.

\begin{figure}[h]
  \centering
  \begin{tabular}{l}
    Le modèle widget a lien très étroit avec les composantes système, ce qui \\
    en fait le modèle de plus efficace dans certaines circonstances. \\
    Il souffre malheureusement d'une configuration très complexe et est \\
    impuissant face à l'échec. Le modèle d'infrastructure, constitué de \\
    composants indépendants, est très convoluté, mais cela ne semble pas etre \\
    un facteur de résignation pour de nombreux développeurs. Toutefois, la \\
    configuration est simple et le système est robuste, ce qui compense le \\
    critère négatif de difficulté de compréhension. Pour finir, le modèle \\
    tableau noir avec des composants très faiblement liés présente des problèmes \\
    d'efficacité, mais il reste simple, robuste et très facilement configurable.
    \cite{winograd_architectures_2001} \\
    \em \footnotesize Terry Winograd. Architectures for Context. \\
    \em \footnotesize Human-Computer Interaction, 16(2):401–419,
     December 2001. \\
  \end{tabular}
  \caption{Comparaison des différents modèles de gestion de contexte}
  \label{fig:quote}
\end{figure}
\begin{quotation}
\end{quotation}

\subsection{Représentation du contexte}

\subsection{Interprétation du contexte}

\subsection{Framework conceptuel en couches}

\subsection{Sécurité et confidentialité}

\section{Systèmes et frameworks existants}

\subsection{Technologies de détection}

\subsection{Représentations du contexte}

\subsection{Découverte des ressouces}

\subsection{Gestion du contexte historique}

\subsection{Sécurité et confidentialité}

\subsection{Conclusion}

% \begin{figure}
% \begin{verbatim}
% #include <iostream>
% int main(int argc, char** argv) {
%   std::cout << "Hello World." << std::endl;
%   return 0;
% }
% \end{verbatim}
%   \caption{Example source code.}
%   \label{fig:sourcecode}
% \end{figure}

%%% Local Variables: ***
%%% mode: latex ***
%%% TeX-master: "thesis.tex" ***
%%% End: ***

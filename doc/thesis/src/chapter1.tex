\chapter{État de l'art}

\section{Introduction}

La configuration des composantes logicielles impose un coût majeur dans
l'administration d'un système. Des erreurs de configuration peuvent se traduire
par des vulnérabilités en termes de sécurité, de sévères pertubations dans le
fonctionnement de la brique logicielle, ou purement et simplement provoquer un
déni de service. La prise en considération du contexte pourrait permettre une
abstraction partielle ou complète de cette couche très technique et extrêmement
pénible à configurer.

Un système sensible au contexte doit être capable de mimer la capacité humaine à
reconaître et exploiter l'information implicitement présente dans
l'environement. Cela implique une configuration dynamique de chacune des
composantes de l'architecture, de manière à pouvoir ajuster leur comportement
respectif en fonction de la situation. Identifier l'activité humaine est un
défi, il est essentiel que les applications opèrent en transmettant
l'information appropriée au bon endroit et au bon moment par inférence de
l'intention des utilisateurs. L'infomatique sensible au contexte est un
paradigme dans lequel les application peuvent découvrir et tirer profit d'
informations de circonstance telles que la position actuelle, l'heure de la
journée, les personnes et périphériques dans l'environement et leurs activités.

Dans ce mémoire, nous aborderons les principes communs à chacunes des
architectures existantes, desquels nous detaillerons le framework conceptuel
dérivé (!) par couches. Nous présenterons un certaine variété d'intergiciels et
d'infrastructures reconnus pour faciliter la configuration d'applications et de
services basés sur le contexte.

\section{Background}

De nombreux débats ont eu lieu sur .. Alors que la plupart des gens comprènent
de manière tacite ce qu'est le contexte, ils le trouve par ailleurs
particulièrement difficile à élucider.

... toute information pouvant être utilisée pour caractériser la situation d'une
entité. Une entité peut être une personne, un lieu ou un objet considéré
pertinent dans l'interaction entre un utilisateur et une application, notamment
l'utilisateur et l'application eux même.

Cette définition rend la tâche plus facile à un développeur d'application pour
énumérer le contexte pour un scénario d'application donné. Si un fragment
d'information peut être utilisé pour caractériser la situation d'un participant
dans une quelquonque interaction, alors cette information appartient au
contexte.

\section{Vue d'ensemble sur le contexte}

Le contexte est efficace, seulement lorsqu'il est partagé.

Pour s'assurer que le contexte soit partagé, il doit d'abord être receilli et
rigoureusement traité. Cela implique qu'un système sensible au contexte doit
être en mesure de comprendre ce qu'est le contexte avant d'aller à la recherche
de ces informations et de pouvoir les catégoriser.

\subsection{Classes de contexte}

Schilit et. al. proposent la classification suivante des informations de
contexte:

\begin{itemize}
  \item \textbf{Contexte Informatique} - Connectivité réseau, bande passante,
	  and ressources à proximité telles que des imprimantes, des
	  affichages ou des postes de travail.
  \item \textbf{Contexte Utilistateur} - Le profil utilisateur, sa situation
	  géographique, sa situation sociale actuelle et les individus qui
	  l'entourent.
  \item \textbf{Contexte Physique} - L'éclairage, le niveau de bruit, les
	  condition de circulation ou la température.
\end{itemize}

Chacune de ces catégories contiennent une richesse d'informations pertinantes
pour le système sensible au contexte. Elle ne peuvent cependant pas être
traitées de manière isolée pour pouvoir en extraire le meilleur. L'intention du
système sensible au contexte est de rassembler et de fusionner ces informations
pour aboutir à une vue d'ensemble de la situation. Une fois le contexte mis en
tampon ou en base, le système doit alors filtrer les informations pertinantes
pour l'utilisateur, dans le moment présent.

Les informations de contexte peuvent alternativement être subdivisées en 2
catégories bien distinctes: contexte virtuel ou physique.

\subsubsection{Contexte virtuel}

Le contexte virtuel inclut la version du système d'exploitation, les
possibilités d'interface, la technologie en charge de l'accomplissement des
communications, les emails envoyés et reçus, et les documents édités.

\subsubsection{Contexte physique}

Les contexte physique d'un autre coté peut être la présence d'une autre entité,
qu'elle soit utilisateur ou périphérique, la proximité d'un imprimante en
particulier, une indication que l'utilisateur est debout, en train de marcher ou
assis ou les conditions météorologiques actuelles. En d'autres termes, le
contexte physique peut être défini comme toute donnée aquierable par le biais
d'une sonde.

\subsubsection{Contexte historique}

Les contextes mémorisés au cours d'un certain laps de temps. Cette information
est considérée très utile, mais n'est que très rarement utilisée, sauf pour les
applications mobiles. Le système doit être en mesure d'estimer les informations
valant la peine d'être conservées. Cette évaluation est excessivement coûteuse
et nécéssite donc des algorithmes très performants.

\subsection{Les caractéristiques des informations de contexte}

\subsection{Système sensible au contexte}

\subsection{Recceuillir l'information}

\subsection{Récupérer l'information}

\subsubsection{Capteur en accès direct}

\subsubsection{Infrastructure intergicielle}

\subsubsection{Serveur de contexte}

\subsection{Les achritectures de gestion de contexte}

\subsubsection{Critères d'arbitrages}

\subsection{Représentation du contexte}

\subsection{Interprétation du contexte}

\subsection{Framework conceptuel en couches}

\subsection{Sécurité et confidentialité}

\section{Systèmes et frameworks existants}

\subsection{Technologies de détection}

\subsection{Représentations du contexte}

\subsection{Découverte des ressouces}

\subsection{Gestion du contexte historique}

\subsection{Sécurité et confidentialité}

\subsection{Conclusion}

% This is an example using the \LaTeX{} template for UCI theses and
% dissertation documents \cite{uci-thesis-latex}. Figure
% \ref{fig:sourcecode} is just for illustration purposes, as is Table
% \ref{tab:coordinates}.
% 
% \begin{figure}
% \begin{verbatim}
% #include <iostream>
% int main(int argc, char** argv) {
%   std::cout << "Hello World." << std::endl;
%   return 0;
% }
% \end{verbatim}
%   \caption{Example source code.}
%   \label{fig:sourcecode}
% \end{figure}

\section{Background}

% Lorem ipsum dolor sit amet, consectetur adipisicing elit, sed do
% eiusmod tempor incididunt ut labore et dolore magna aliqua. Ut enim ad
% minim veniam, quis nostrud exercitation ullamco laboris nisi ut
% aliquip ex ea commodo consequat. Duis aute irure dolor in
% reprehenderit in voluptate velit esse cillum dolore eu fugiat nulla
% pariatur. Excepteur sint occaecat cupidatat non proident, sunt in
% culpa qui officia deserunt mollit anim id est laborum.
% 
% \begin{table}
%   \centering
%   \begin{tabular}{|rr|r|}
%     \hline
%     $x$ & $y$ & $z$ \\
%     \hline
%     14 & 12 & -2 \\
%     0 & 33 & -25 \\
%     -3 & 11 & 22 \\
%     4 & 4 & 6 \\
%     \hline
%   \end{tabular}
%   \caption{Example coordinates.}
%   \label{tab:coordinates}
% \end{table}
% 
% Lorem ipsum dolor sit amet, consectetur adipisicing elit, sed do
% eiusmod tempor incididunt ut labore et dolore magna aliqua. Ut enim ad
% minim veniam, quis nostrud exercitation ullamco laboris nisi ut
% aliquip ex ea commodo consequat. Duis aute irure dolor in
% reprehenderit in voluptate velit esse cillum dolore eu fugiat nulla
% pariatur. Excepteur sint occaecat cupidatat non proident, sunt in
% culpa qui officia deserunt mollit anim id est laborum.


%%% Local Variables: ***
%%% mode: latex ***
%%% TeX-master: "thesis.tex" ***
%%% End: ***

\chapter{Simulation}

\begin{figure}
  \begin{verbatim}
  #include <iostream>
  int main(int argc, char** argv) {
    std::cout << "Hello World." << std::endl;
    return 0;
  }
  \end{verbatim}
  \caption{Example source code.}
  \label{fig:sourcecode}
\end{figure}

\section{Conclusion}

Component technologies will play a fundamental role in the next generation
computer systems as the complexity of software and the diversity and
pervasiveness of computing devices increase. However, component technologies
must offer mechanisms for automatic management of inter-component dependencies
and component-to-resource dependencies. Otherwise, the development of
component- based systems will continue to be difficult and frequently lead to
unreliable and non-robust systems.  

Future ubiquitous computing environments will be composed of thousands of
devices running millions of software components. Current systems rely heavily
on manual configuration but with a system composed of millions of components
this will no longer be possible.  There are only two ways out of this
situation: static configuration or dynamic, automatic configuration

Since future environments tend to be more and more dynamic, automatic
configuration seems to be the only viable solution.

\chapter{Simulation}

\section{Framework de gestion de configuration}

La gestion de la configuration est le processus de contraindre le comportement
d'un réseau de machines de manière à ce que le comportement de chaque machine
soit conforme aux politiques et les lignes directrices prédéfinies, et
accomplisse les objectifs d'entreprise prédéterminés.

Cela implique :

\begin{itemize}
  \item Des personnes
  \item Outil gestion de configuration zéro (ex: CfEngine)
  \item Un ensemble ce machines interconnectées
  \item Un ensemble de processus de configuration destiné à aboutir à un système
	  conforme à la politique en vigueur.
\end{itemize}

Les paramètre de configuration peuvent être la permission d'un fichier,
l'adresse d'un carte réseau, le type de système de fichier pouvant être monté
sur un volume disque.

La possibilité d'automatiser la configuration des réseaux IP (Internet
Protocol) en utilisant une technologie sémantique est également très
prometteuse. L'initiative connexe de créer un environnement de réseau de
capteurs basé sur le ontologies semble également etres une solution viable.

La plupart des organisations aujourd'hui n'ont aucun procédé systématique pour
gérer leurs informations de configuration. Celà signifie que des informations
comme l'adresse IP d'une machine et le statut de son service réseau sont
stockées de manière dispersée et désorganisée. Certains outils de supervision
tels que Nagios présente une manière de régorganiser ces informations provenant
de source dispersées.

L'idée d'utiliser CMDB est également un alternative permettant de solver le
problème ci-dessus. Toutefois, le besoin pour un gestionnaire d'informations de
plus haut niveau est mentionné dans plusieures littératures.

Cette nouvelle façon de renforcer l'efficacité de la gestion des connaissances
dans le domaine de la gestion de configuration, nécessite la possibilité
d'extraction automatique de données à partir d'emplacements de stockage
précédents et la possibilité de mise à jour automatique de stockages permanents
tels que la CMDB afin d'augmenter la qualité de l'information sous-jacente.

La possibilité de ces deux fonctions à savoir, l'extraction automatique des
données et également la mise à jour automatique des informations stockées ont
été fournies par ITIL.

Le travail de ce mémoire est e démontrer les possibilité d'obtenir un
gestionnaire de conaissances intégré en conjonguant les ontologies, la théorie
de la promesse et le consensus de Raft. La structure de l'information de
configuration stockée dans une CMDB and d'autre documents sera réprésentée en
concordance avec les standard de OWL. Cette struture de domaine de connaissance
sera implémentée comme une promesse de structure d'information en utilisant
CfEngine3 (ou pas).

\noindent\rule{8cm}{0.4pt}

Note: A configuration management database (CMDB) is a repository that acts as a
data warehouse for information technology (IT) organizations

La Configuration Management DataBase (abrégé CMDB), ou base de données de
gestion de configuration, est une base de données unifiant les composants d'un
système informatique. Elle permet de comprendre l'organisation entre ceux-ci et
de modifier leur configuration. La CMDB est un composant fondamental d'une
architecture ITIL.

\section{Ontologie pour un gestionnaire de configuration basé sur la théorie de
la promesse}


\begin{figure}
  \begin{verbatim}
  #include <iostream>
  int main(int argc, char** argv) {
    std::cout << "Hello World." << std::endl;
    return 0;
  }
  \end{verbatim}
  \caption{Example source code.}
  \label{fig:sourcecode}
\end{figure}

\section{Conclusion}

Component technologies will play a fundamental role in the next generation
computer systems as the complexity of software and the diversity and
pervasiveness of computing devices increase. However, component technologies
must offer mechanisms for automatic management of inter-component dependencies
and component-to-resource dependencies. Otherwise, the development of
component- based systems will continue to be difficult and frequently lead to
unreliable and non-robust systems.  

Future ubiquitous computing environments will be composed of thousands of
devices running millions of software components. Current systems rely heavily
on manual configuration but with a system composed of millions of components
this will no longer be possible.  There are only two ways out of this
situation: static configuration or dynamic, automatic configuration

Since future environments tend to be more and more dynamic, automatic
configuration seems to be the only viable solution.
